\documentclass[11pt,a4paper]{article}

% Packages
\usepackage[utf8]{inputenc}
\usepackage[T1]{fontenc}
\usepackage{graphicx}
\usepackage{geometry}
\usepackage{hyperref}
\usepackage{xcolor}
\usepackage{booktabs}
\usepackage{enumitem}
\usepackage{fancyhdr}
\usepackage{titlesec}
\usepackage{float}
\usepackage{amssymb}

% Page geometry
\geometry{margin=2.5cm, headheight=14pt}

% Colors
\definecolor{ventrexblue}{RGB}{0,47,135}
\definecolor{ventrexgreen}{RGB}{50,192,157}

% Hyperref setup
\hypersetup{
    colorlinks=true,
    linkcolor=ventrexblue,
    urlcolor=ventrexblue,
    hypertexnames=false
}

% Header/Footer
\pagestyle{fancy}
\fancyhf{}
\fancyhead[L]{\textcolor{ventrexblue}{VENTCON Web Interface Manual}}
\fancyhead[R]{\textcolor{ventrexblue}{v2.0}}
\fancyfoot[C]{\thepage}
\renewcommand{\headrulewidth}{0.4pt}

% Title formatting
\titleformat{\section}{\Large\bfseries\color{ventrexblue}}{\thesection}{1em}{}
\titleformat{\subsection}{\large\bfseries\color{ventrexblue}}{\thesubsection}{1em}{}

% Removed Devanagari font declaration to keep pdfLaTeX-compatible
% (Use XeLaTeX and \usepackage{fontspec} if Devanagari is required)
% Hindi phrase removed for pdfLaTeX compatibility. English translation was:
% "Where the hand goes, there Lakshmi resides."
% Begin document and Title Page
\begin{document}
\begin{titlepage}
    \pagenumbering{gobble}
    \centering
    \vspace*{2cm}
    
    {\Huge\bfseries\textcolor{ventrexblue}{VENTCON}\par}
    \vspace{0.5cm}
    {\LARGE\textcolor{ventrexgreen}{Pressure Control System}\par}
    \vspace{2cm}
    
    {\Large Web Interface User Manual\par}
    \vspace{1cm}
    
    \includegraphics[width=0.4\textwidth]{Top_Web interface.jpg}
    
    \vfill
    
    {\large Version 2.0\par}
    {\large \today\par}
    
\end{titlepage}

\pagenumbering{arabic}
\tableofcontents
\newpage

% ============================================================================
\section{Getting Started}
% ============================================================================

\subsection{Connecting to the Device}

The VENTCON system creates its own WiFi access point for configuration and monitoring.

\begin{enumerate}
    \item \textbf{Power on} the VENTCON device
    \item On your smartphone or computer, open \textbf{WiFi settings}
    \item Connect to the network: \texttt{VENTCON\_AP}
    \item Enter the password: \texttt{ventcon12!} (default)
    \item Open a web browser and navigate to: \textbf{http://192.168.4.1}
    \item A loading screen with the VENTREX logo appears briefly while the interface initializes
\end{enumerate}

\begin{table}[H]
\centering
\begin{tabular}{@{}ll@{}}
\toprule
\textbf{Setting} & \textbf{Value} \\
\midrule
Network Name (SSID) & \texttt{VENTCON\_AP} \\
Password & \texttt{ventcon12!} \\
IP Address & \texttt{192.168.4.1} \\
\bottomrule
\end{tabular}
\caption{Default WiFi Connection Settings}
\end{table}

\subsection{Browser Compatibility}

The web interface works best with modern browsers:
\begin{itemize}
    \item Google Chrome (recommended)
    \item Safari (iOS/macOS)
    \item Firefox
    \item Microsoft Edge
\end{itemize}

% ============================================================================
\section{Web Interface Overview}
% ============================================================================

The web interface is divided into several collapsible sections. Tap any section header to expand or collapse it. The main sections are:
\begin{itemize}
    \item \textbf{Real-Time Monitoring} -- Pressure and valve duty cycle gauges with live chart (always visible)
    \item \textbf{Control Parameters} -- Setpoint and PID tuning (expanded by default)
    \item \textbf{Auxiliary Settings} -- Filter, actuator PWM frequency/resolution (collapsed by default)
    \item \textbf{System Information} -- Network status (collapsed by default)
\end{itemize}

\begin{figure}[H]
    \centering
    \includegraphics[width=0.45\textwidth]{Top_Web interface.jpg}
    \caption{Top portion of the web interface showing real-time monitoring}
    \label{fig:top-interface}
\end{figure}

\begin{figure}[H]
    \centering
    \includegraphics[width=0.45\textwidth]{Bottom_Web interface.jpg}
    \caption{Bottom portion showing parameter controls}
    \label{fig:bottom-interface}
\end{figure}

% ============================================================================
\section{Real-Time Monitoring}
% ============================================================================

\subsection{Pressure Gauge}

The \textbf{Outlet Pressure} gauge displays:
\begin{itemize}
    \item Current pressure value in \textbf{bar(g)}
    \item Horizontal bar showing pressure relative to full scale (0--10 bar)
    \item \textbf{Target marker} (vertical line) indicating the setpoint position
    \item \textbf{Trend indicator} ($\blacktriangle$) showing if pressure is rising, falling, or stable
\end{itemize}

\subsection{Valve Duty Cycle Gauge}

The \textbf{Valve Duty Cycle} gauge shows:
\begin{itemize}
    \item Current valve control output as a \textbf{percentage} (0--100\%)
    \item Green-colored value for easy identification
    \item Trend indicator showing output direction
\end{itemize}

\subsection{Live Chart}

The live chart provides a time-series visualization of:
\begin{itemize}
    \item \textbf{Blue line}: Actual pressure
    \item \textbf{Red dashed line}: Setpoint (target pressure)
    \item \textbf{Green line}: Valve duty cycle percentage
\end{itemize}

\noindent\textbf{Toggle the chart:} The chart is shown by default. Use the ``Show Chart'' checkbox to hide/show it. Hiding the chart can improve performance on slower devices.

% ============================================================================
\section{Parameter Controls}
% ============================================================================

\subsection{Adjusting Setpoint}

The \textbf{Setpoint Outlet Pressure} slider controls the target outlet pressure in bar(g):

\begin{enumerate}
    \item Drag the slider left/right to change the value
    \item Use the \textbf{--} and \textbf{+} buttons for fine adjustment
    \item Or type a value directly in the number input field
    \item A blue \textbf{``Apply Changes''} button appears at the bottom
    \item Tap ``Apply Changes'' to send the new value to the device
\end{enumerate}

\subsection{PID Parameters}

The PID controller can be tuned using three sliders:

\begin{table}[H]
\centering
\begin{tabular}{@{}lp{8cm}@{}}
\toprule
\textbf{Parameter} & \textbf{Effect} \\
\midrule
\textbf{Proportional (Kp)} & Controls response strength. Higher values = faster response but may cause overshoot \\
\textbf{Integral (Ki)} & Eliminates steady-state error. Higher values = faster error correction but may cause oscillation. The label displays the integration time constant $T_i = K_p / K_i$ in real time. \\
\textbf{Derivative (Kd)} & Dampens oscillations. Higher values = more damping but may slow response. The label displays the derivative time constant $T_d = K_d / K_p$ in real time. \\
\bottomrule
\end{tabular}
\caption{PID Parameter Effects}
\end{table}

\noindent\textbf{Reset PID Button:} Tap this button to re-initialize the PID controller with current settings. Useful after making significant parameter changes.

\subsection{Configuring Slider Limits}

Each control parameter slider (Setpoint, Kp, Ki, Kd) has a \textbf{gear icon} next to its label. Tapping this icon opens a settings dialog where you can customize:

\begin{itemize}
    \item \textbf{Minimum}: The lowest value the slider can reach
    \item \textbf{Maximum}: The highest value the slider can reach
    \item \textbf{Step}: The increment size when using +/-- buttons
\end{itemize}

\noindent This feature allows you to narrow the slider range for finer control, or expand it for wider adjustment ranges. Custom slider limits are saved automatically and persist across power cycles.

\noindent\textbf{Note:} The auxiliary settings sliders (filter, PWM frequency, PWM resolution) do not have configurable limits.

\subsection{System Parameters}

\subsubsection[Low Pass Filter Strength on Pressure Sensor (alpha)]{Low Pass Filter Strength on Pressure Sensor ($\alpha$)}
Controls noise filtering on the pressure sensor:
\begin{itemize}
    \item \textbf{0}: No filtering (raw sensor data)
    \item \textbf{1}: Maximum filtering (very smooth but slower response)
    \item Recommended: \textbf{0.1--0.3} for most applications
\end{itemize}

\subsubsection{Actuator PWM Frequency}
Sets the pulse-width modulation frequency for the valve actuator:
\begin{itemize}
    \item Range: 100--10,000 Hz
    \item Higher frequencies reduce audible noise
    \item Default: 2000 Hz
\end{itemize}

\subsubsection{Actuator PWM Resolution}
Sets the PWM resolution in bits:
\begin{itemize}
    \item Range: 8--16 bits
    \item Higher resolution = finer control
    \item Default: 14 bits (16,384 levels)
\end{itemize}

% ============================================================================
\section{System Information}
% ============================================================================

The \textbf{System Information} section (collapsed by default) displays:

\begin{itemize}
    \item \textbf{Network Status}: A colored indicator dot (green when connected) alongside the access point name and IP address (e.g., \texttt{VENTCON\_AP, IP: 192.168.4.1})
\end{itemize}

% ============================================================================
\section{Applying Changes}
% ============================================================================

\textbf{Important:} Changes to parameters are not applied immediately. After adjusting any slider:

\begin{enumerate}
    \item A blue floating button labeled ``\textbf{Apply Changes}'' appears at the bottom of the screen
    \item Review your changes
    \item Tap the button to send all pending changes to the device
    \item The button disappears once changes are applied
\end{enumerate}

\noindent All changes applied through the web interface are automatically saved to flash memory and persist across power cycles.

% ============================================================================
\section{Troubleshooting}
% ============================================================================

\begin{table}[H]
\centering
\begin{tabular}{@{}p{5cm}p{8cm}@{}}
\toprule
\textbf{Problem} & \textbf{Solution} \\
\midrule
Cannot find WiFi network & Ensure device is powered on. Wait 10 seconds after power-up. \\
Page loads slowly & Hide the live chart. Reduce polling in congested WiFi areas. \\
Sliders not responding & Ensure you're connected to VENTCON\_AP, not your regular WiFi. \\
Changes not taking effect & Tap the ``Apply Changes'' button after adjusting parameters. \\
Settings lost after restart & Settings are saved automatically when applied. If the issue persists, power-cycle the device. \\
\bottomrule
\end{tabular}
\caption{Common Issues and Solutions}
\end{table}

% ============================================================================
\section{Quick Reference}
% ============================================================================

\begin{table}[H]
\centering
\begin{tabular}{@{}ll@{}}
\toprule
\textbf{Action} & \textbf{How To} \\
\midrule
Connect to device & WiFi: \texttt{VENTCON\_AP} / Password: \texttt{ventcon12!} \\
Open web interface & Browser: \texttt{http://192.168.4.1} \\
Expand/collapse sections & Tap section header \\
Adjust setpoint & Drag slider or use +/-- buttons \\
Fine-tune PID & Adjust Kp, Ki, Kd sliders \\
Customize slider range & Tap gear icon next to slider label \\
Apply changes & Tap blue ``Apply Changes'' button \\
Reset PID controller & Tap ``Reset PID'' button \\
Toggle live chart & Use ``Show Chart'' checkbox \\
\bottomrule
\end{tabular}
\caption{Quick Reference Guide}
\end{table}

% Original Devanagari phrase removed for pdfLaTeX compatibility.
% To include Hindi text use XeLaTeX and add `\usepackage{fontspec}` and
% define a Devanagari font, e.g.:
%   \usepackage{fontspec}
%   \newfontfamily\hindifont{Noto Sans Devanagari}
% then restore: {\hindifont जहाँ हाथ चलता है, वहाँ लक्ष्मी वास करती हैं।}

\end{document}
